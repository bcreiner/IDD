\documentclass[fancyhdr,10pt]{article}
\usepackage{amsfonts}
\usepackage{amsmath}
\usepackage{amssymb}
\usepackage{anysize}
\usepackage{watermark}
\usepackage{fancyhdr}           % Headers and footers
\pagestyle{fancy}
\pdfpageheight=\paperheight
\pdfpagewidth=\paperwidth
%\usepackage[pdftex]{hyperref}
%\hypersetup{colorlinks=true, citecolor=black, urlcolor=black, pdfpagemode=none}
%\usepackage[colorhighlight,display,whitebackground]{texpower}
\usepackage{color}
\usepackage{subfigure}
\usepackage{calc}
\usepackage{array}
\usepackage{lastpage}
\usepackage{enumerate}
\usepackage{graphicx}           % Standard graphics package
\usepackage{mathrsfs}
\usepackage{wrapfig}
\usepackage{dsfont}
\usepackage{multirow}
\usepackage{rotating}
\usepackage{multicol}
\usepackage{xcolor}
\usepackage{hyperref}
\hypersetup{
    colorlinks,
    linkcolor={red!50!black},
    citecolor={blue!50!black},
    urlcolor={blue!80!black}
}
\usepackage{palatino}%newcent palatino palatcm
\newcommand{\gargantuan}{\fontsize{120}{60}\selectfont}
\newcommand{\HUGE}{\fontsize{30}{15}\selectfont}
  \fancyhf{}
%  \renewcommand{\headrule}{}
%  \renewcommand{\footrule}{}
  \renewcommand{\footrulewidth}{0pt}
  \renewcommand{\headrulewidth}{1.5pt}
  % Headers and footers personalization using the `fancyhdr' package
  \fancyhead[L]{HMS ???}
  \fancyhead[R]{Spring ???}
  % \fancyhead[C]{\textsc{Advanced Methods For Global Health II}}
%  \fancyhead[L]{\usebox{\mylogo}}
%\fancyhead[R]{\usebox{\includegraphics[scale=.675]{CSUNCSMlogo.jpg}}}
%  \fancyfoot[L]{\includegraphics[width=1in]{UMLogo.gif}}
  \fancyfoot[R]{}
 \pagestyle{fancy}
\usepackage{tikz}

\setlength{\topmargin}{-1in}
\setlength{\headheight}{.5in}
\setlength{\textheight}{10in}
\setlength{\evensidemargin}{-.5in}
\setlength{\oddsidemargin}{-.5in}
\setlength{\textwidth}{7.25in}
\setlength{\headsep}{.25in}
\setlength{\topskip}{.5in}
\setlength{\footskip}{0in}
\setlength{\headwidth}{7.5in}


\newcommand{\LG}{.1in} %Letter Gap
\newcommand{\IG}{.55in} %Infection Gap
\newcommand{\IGG}{.45in} %Infection Gap
\newcommand{\CHG}{.45in} %ClusterFuck Height Gap
\newcommand{\CWG}{.65in} %ClusterFuck Width Gap
\newcommand{\RG}{1.5in} %Row gap
\newcommand{\BG}{1in} %Big gap
\newcommand{\TG}{.6in} %Third infection middle gap
\newcommand{\PSI}[2]{$\psi_{#2|#1}$}
\newcommand{\Com}[4]{$#1_i\ #2_j\ #3_k\ #4_l$}
\newcommand{\Mul}[1]{$\beta_{#1}$}


\usetikzlibrary{arrows,positioning,fit,calc}
\tikzstyle{line} = [draw, -latex']
\tikzset{
  label/.style = {font=\footnotesize, midway, fill=white, anchor=center}
}
\renewcommand{\arraystretch}{1.4}
\setlength\parindent{0pt}
\begin{document}
\hbox{}\vspace{-.6in}
\begin{center}
\LARGE\textsc{Infectious disease dynamics: Models and data}
\end{center}


\section*{\underline{\textsc{Class Information}}}


  \begin{tabular}{m{3.5in} m{4.5in}}
Instructor: Dr. Bobby Reiner & Teaching Assistant: ???\\
Class Meetings: ??? & Lab Meetings: ???\\
Email: bcreiner@uw.edu & ???\\
Office Hours: ???\\
\end{tabular}

\renewcommand{\arraystretch}{1.2}
\section*{\underline{\textsc{Course description and objectives}}}
This course provides an overview of various infectious disease models as well as approaches to fit these models to data. We will cover foundational underlying mathematics, issues with incomplete and biased data, as well as model selection considerations associated with spatial and temporal scale. Assumes prior knowledge of the fundamentals of matrix / linear algebra, differential equations, and some exposure to compartmental models.\\

\noindent Learning objectives:\\
\begin{enumerate}
 \item Describe the appropriate use case for a variety of infectious disease models
 \item Identify and adjust for data inconsistencies, biases, or levels of incompleteness
 \item Implement a variety of model fitting techniques
 \item Identify issues of spatial and temporal scales of transmission and how they inform model selection
 \item Diagnosis issues with model fitting / model forecasting associated with (1) model assumptions, (2) data assumptions, and (3) model fitting assumptions
\end{enumerate}


\section*{\underline{\textsc{Resources}}}
All required resources will be posted on Canvas: To come\\

Recommended reading: Bj\o rnstad (2018). Epidemics: Models and Data Using R. Springer.

\section*{\underline{\textsc{Course prerequisite}}}
Required - A course in matrix algebra / linear algebra or permission of the instructor; a course in differential equations or permission of the instructor; previous familiarity with Python or R (no exceptions)\\
Optional - EPI 554; AMATH 502; AMATH 535; STAT 491


\section*{\underline{\textsc{Requirements and expectations}}}

\textit{Participation}. Students are expected to attend course lectures and lab sections, and actively participate in class discussions. Participation will account for 20$\%$ of the final grade total.\\

\textit{Reading}. Students are expected to complete the required readings by the date listed in the course syllabus and come prepared to discuss and apply the material covered in the readings.\\

\textit{Individual Assignments}. Students will complete 4 individual assignments in which they will apply the concepts and methods taught in the preceeding weeks. Each assignment will be distributed by the instructor the week before it is due, submitted before class on the due date using Canvas and worth 20$\%$ of the final grade. (20$\%$ x 4 = 80$\%$ of final grade total).\\

\section*{\underline{\textsc{Summary of course grading}}}
Participation: worth 20$\%$ of grade. Individual assingments (4): Each worth 20$\%$ of grade.

\section*{\underline{\textsc{Class schedule}}}
\textit{Lectures} Course lectures will be 1 hour and 20 minutes and occur on ???.\\

\textit{TA-led Lab section}. There will be a required 1-hour TA-led lab session on ???. In this session the TA will go over models and code discussed during lecture and assist in the implementation of addition code assocaited with model development or model fitting. The TA will also discuss answers to the homework submitted on ???, answer any questions on the upcoming homework assignment. The TA will focus on R and Python coding issues as well.

\section*{\underline{\textsc{Statistical software:} \texttt{R} or Python}}
Lectures will demonstrate material in \verb!R! and Python. Assignments are expected to be complted in either \verb!R! or Python (no exceptions). More detail on homework structure will be discussed in the course. \verb!R! is freely available from \href{http://cran.r-project.org/}{http://cran.r-project.org/} and Python can be freely downloaded from \href{https://www.python.org/downloads/release/python-3100/}.

\section*{\underline{\textsc{Academic Integrity Statement}}}

Students at the University of Washington (UW) are expected to maintain the highest standards of academic conduct, professional honesty, and personal integrity. The UW School of Public Health (SPH) is committed to upholding standards of academic integrity consistent with the academic and professional communities of which it is a part. Plagiarism, cheating, and other misconduct are serious violations of the University of Washington Student Conduct Code (WAC 478-120). We expect you to know and follow the university's policies on cheating and plagiarism, and the \href{https://sph.washington.edu/students/academic-integrity-policy}{SPH Academic Integrity Policy}. Any suspected cases of academic misconduct will be handled according to University of Washington regulations. For more information, see the University of Washington Community Standards and Student Conduct website.



\section*{\underline{\textsc{Access and Accommodations}}}
Your experience in this class is important to me. If you have already established accommodations with Disability Resources for Students (DRS), please communicate your approved accommodations to me at your earliest convenience so we can discuss your needs in this course.\\

If you have not yet established services through DRS, but have a temporary health condition or permanent disability that requires accommodations (conditions include but not limited to; mental health, attention-related, learning, vision, hearing, physical or health impacts), you are welcome to contact DRS at 206-543-8924 or uwdrs@uw.edu or disability.uw.edu. DRS offers resources and coordinates reasonable accommodations for students with disabilities and/or temporary health conditions.  Reasonable accommodations are established through an interactive process between you, your instructor(s) and DRS.  It is the policy and practice of the University of Washington to create inclusive and accessible learning environments consistent with federal and state law.


\section*{\underline{\textsc{Diversity statement}}}
Diverse backgrounds, embodiments, and experiences are essential to the critical thinking endeavor at the heart of university education. Therefore, I expect you to follow the UW Student Conduct Code in your interactions with your colleagues and me in this course by respecting the many social and cultural differences among us, which may include, but are not limited to: age, cultural background, disability, ethnicity, family status, gender identity and presentation, citizenship and immigration status, national origin, race, religious and political beliefs, sex, sexual orientation, socioeconomic status, and veteran status. Please talk with me right away if you experience disrespect in this class, and I will work to address it in an educational manner. DCinfo@uw.edu is a resource for students with classroom climate concerns.\\

Required Syllabus Language: ``Washington state law requires that UW develop a policy for accommodation of student absences or significant hardship due to reasons of faith or conscience, or for organized religious activities. The UW's policy, including more information about how to request an accommodation, is available at \href{https://registrar.washington.edu/staffandfaculty/religious-accommodations-policy/}{Religious Accommodations Policy}. Accommodations must be requested within the first two weeks of this course using the \href{https://registrar.washington.edu/students/religious-accommodations-request/}{Religious Accommodations Request form}.''




\newpage
\section*{\underline{\textsc{Tentative Course Outline}}}
The subject matter covered and the corresponding dates listed below are subject to change. Design and Analysis sessions will frequently continue into the poseceding case study session.  Please read the posted reading before class.\\

\begin{center}
  \begin{tabular}{|c|c|m{3in}|l|}
    \hline
    % after \\: \hline or \cline{col1-col2} \cline{col3-col4} ...
    Week                & Date  & Topic & Assignments \\\hline\hline
    \multirow{3}{*}{1}  & ?  & Course Introduction &  \\\cline{2-4}
                        & ?  & Lab - RMarksown and Jupyter Notebooks & \\\cline{2-4}
                        & ?  & Catalytic model / FoI by age $\&$ time &\\\hline\hline
    \multirow{3}{*}{2}  & ?  & $\beta \cdot S \cdot I$ &  HW 1  \\\cline{2-4}
                        & ?  & Lab - Fitting FoI to data / trajectory matching & \\\cline{2-4}
                        & ?  & Metrics of transmission &\\\hline\hline
    \multirow{3}{*}{3}  & ?  & Stochasticity & HW 1 Due \\\cline{2-4}
                        & ?  & Lab - Fitting stochastic SIR $\&$ time & \\\cline{2-4}
                        & ?  & TSIR models &\\\hline\hline
    \multirow{3}{*}{4}  & ?  & Spill-over events & HW 2  \\\cline{2-4}
                        & ?  & Lab - TSIR models and stuttering chains & \\\cline{2-4}
                        & ?  & Advanced fitting techniques &\\\hline\hline
    \multirow{3}{*}{5}  & ?  & Data considerations &  HW 2 Due   \\\cline{2-4}
                        & ?  & Lab - Fitting with partial observations, POMP & \\\cline{2-4}
                        & ?  & Observation process &\\\hline\hline
    \multirow{3}{*}{6}  & ?  & Special topics - COVID & HW 3  \\\cline{2-4}
                        & ?  & Lab - Fitting COVID models & \\\cline{2-4}
                        & ?  & Age structure / mixing &\\\hline\hline
    \multirow{3}{*}{7}  & ?  & Network models & HW 3 Due \\\cline{2-4}
                        & ?  & Differential mixing models & \\\cline{2-4}
                        & ?  & Spatial models - discrete &\\\hline\hline
    \multirow{3}{*}{8}  & ?  & Spatial models - (semi-)continuous & \\\cline{2-4}
                        & ?  & Lab - Patch models & \\\cline{2-4}
                        & ?  & Scales of transmisison &\\\hline\hline
    \multirow{3}{*}{9}  & ?  & Multi-host models - malaria & HW 4\\\cline{2-4}
                        & ?  & Lab - Dengue versus malaria & \\\cline{2-4}
                        & ?  & Agent-based models &\\\hline\hline
    \multirow{3}{*}{10} & ?  & Special topics - Forecasting  & HW 4 Due \\\cline{2-4}
                        & ?  & Lab - ABMs versus Multi-patch versus single well-mixed & \\\cline{2-4}
                        & ?  & Special topics - Multiple diseases  &\\\hline\hline
  \end{tabular}
\end{center}

\section*{\underline{\textsc{Key dates}}}
\begin{itemize}
 \item Start of second week - Homework 1 distributed
 \item Start of third week - Homework 1 due
 \item Start of fourth week - Homework 2 distributed
 \item Start of fifth week - Homework 2 due
 \item Start of sixth week - Homework 3 distributed
 \item Start of seventh week - Homework 3 due
 \item Start of ninth week - Homework 3 distributed
 \item Start of tenth week - Homework 3 due
\end{itemize}


\end{document}
